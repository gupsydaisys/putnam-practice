\documentclass[11pt]{article}
\usepackage{amsmath,textcomp,amssymb,geometry,graphicx,enumerate}
\usepackage{algorithmicx}
\usepackage[ruled]{algorithm}
\usepackage{algpseudocode}
\usepackage{algpascal}
\usepackage{algc}
\usepackage{tkz-graph}
\usepackage{verbatim}

\def\Name{Serena Gupta}  % Your name
\def\SID{22830625}  % Your student ID number
\def\Homework{11}%Number of Homework
\def\Session{Fall 2015}

\title{MATH 191 --- Fall 2015 --- Homework \Homework\ Solutions}
\author{\Name, \SID}
\markboth{MATH 191 --\Session\  Homework \Homework\ \Name}{\Name,\ \SID\ -------- Math 191 Problem Set \Homework}
\pagestyle{myheadings}

\newenvironment{qparts}{\begin{enumerate}[{(}a{)}]}{\end{enumerate}}
\def\endproof{\text{  } \square}
\newcommand{\p}[1]{\left(#1\right)}
\renewcommand{\b}[1]{\left[#1\right]}
\newcommand{\floor}[1]{\left\lfloor#1\right\rfloor}
\newcommand{\ceil}[1]{\left\lceil#1\right\rceil}
\newcommand{\argmin}{\operatornamewithlimits{argmin}}
\newcommand{\argmax}{\operatornamewithlimits{argmax}}
\newcommand{\mbp}{\mathbb{P}}
\renewcommand{\P}[1]{\mathbb{P}\p{#1}}
\renewcommand{\Pr}{\text{Pr}}
\newcommand{\E}[1]{\mathbb{E}\b{#1}}
\newcommand{\Var}[1]{\mathrm{Var}\p{#1}}
\newcommand{\Cov}[1]{\mathrm{Cov}\p{#1}}
\newcommand{\indep}{\rotatebox[origin=c]{90}{$\models$}}
\newcommand{\F}{\mathcal{F}}

\textheight=9in
\textwidth=6in
\topmargin=-.75in
\oddsidemargin=0.25in
\evensidemargin=0.25in

\begin{document}

$\\$ \textbf{3) } If $n$ is odd, $P(n+1) = 0$.  Otherwise (ie when $n$ is even), $P(n+1) = -1$.

$\\$ \textbf{Proof: } Using Lagrange Interpolation, we get that:
\begin{align*}
P(x) = \sum_{k=0}^{n}\binom{n+1}{k}^{-1}\prod_{j=0 \text{ st } j \neq k}^{n}\frac{(x - j)}{(k - j)} \\
\end{align*}

$\\$ Thus putting as input $(n+1)$, we get:
\begin{align*}
P(n+1) &= \sum_{k=0}^{n}\binom{n+1}{k}^{-1}\prod_{j=0 \text{ st } j \neq k}^{n}\frac{(n + 1 - j)}{(k - j)} \\
&= \sum_{k=0}^{n}\frac{(k)!(n+1 - k)!}{(n+1)!}\frac{\prod\limits_{i=n}^{k+1}(n + 1 - i)\prod\limits_{j=k-1}^{1}(n + 1 - j)}{\prod\limits_{i=n}^{k+1}(k - i)\prod\limits_{j=k-1}^{1}(k - j)} \\
&= \sum_{k=0}^{n}\frac{(k)!(n+1 - k)!}{(n+1)!}\frac{(n-k)!\prod\limits_{j=k-1}^{1}(n + 1 - j)}{\prod\limits_{i=n}^{k+1}(k - i)\prod\limits_{j=k-1}^{1}(k - j)} \\
&= \sum_{k=0}^{n}\frac{(k)!(n+1 - k)!}{(n+1)!}\frac{(n-k)!\frac{(n+1)!}{(n+1-k)!}}{\prod\limits_{i=n}^{k+1}(k - i)\prod\limits_{j=k-1}^{1}(k - j)} \\
&= \sum_{k=0}^{n}\frac{(k)!(n+1 - k)!}{(n+1)!}\frac{(n-k)!\frac{(n+1)!}{(n+1-k)!}}{(n-k)!(-1)^{n-k}\prod\limits_{j=k-1}^{1}(k - j)} \\
&= \sum_{k=0}^{n}\frac{(k)!(n+1 - k)!}{(n+1)!}\frac{(n-k)!\frac{(n+1)!}{(n+1-k)!}}{(n-k)!(-1)^{n-k}(k)!} \\
&= \sum_{k=0}^{n}(-1)^{n-k} \\
&= \left\{\begin{matrix}
0 & \text{ if n is odd} \\ 
1  & \text{ o.w.} & 
\end{matrix}\right.\endproof
\end{align*}

\newpage
\textbf{7) } Suppose towards a contradiction, that for any distance d, there are no 2 points of color red or yellow which are d apart.

$\\$ Color a point yellow and draw a circle, $C_y$, whose center is the yellow point and with radius d.

$\\$ Clearly in order for this condition to hold all points on the circumference of the circle must be red.

$\\$ Now, draw another circle, $C_r$, with radius $d$ and whose center point is on the circumference of $C_y$.

$\\$ All the points on the circumference of $C_r$ must be yellow but the circle intersects the $C_y$ so clearly at least 2 points are red.  

$\\$ Thus we have a contradiction and it must be true that for one of the two colors $C$, it is strue that for any distance $d$, there are 2 points of color $C$ which are $d$ apart. $\endproof$

\newpage
\textbf{8) } The answer is 0 but I wasn't able to rigoriously prove why that's true.
 

\newpage
\textbf{9) } I spent probably 12 or so hours on it.  I liked the variety in problems this week!  I thought the matrix problems were hard.

$\\$ Lagrange Interpolation can be used to find the least degree polynomial such that at each point $x_j$ assumes the corresponding value $y_i$.  The form the polynomial actually takes is exemplified in my solution to number 3.

\end{document}